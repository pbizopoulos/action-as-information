\documentclass[twoside,11pt]{article}
\usepackage{jmlr2e}

\usepackage{lastpage}
\jmlrheading{00}{0000}{1-\pageref{LastPage}}{1/21; Revised 0/00}{0/00}{00-0000}{Paschalis Bizopoulos} % chktex 8

\ShortHeadings{Action as Information}{Paschalis Bizopoulos}
\firstpageno{1}


\begin{document}
\title{Action as Information}
\author{\name{Paschalis Bizopoulos} \email{pbizopoulos@protonmail.com} \addr{Thessaloniki, Greece}}

\editor{My editor}

\maketitle

\begin{abstract}
	We define action as proportional to the energy and description length of some data.
	We argue that this definition of action can be used to quantify information and discuss how it relates with previous literature.
\end{abstract}

\begin{keywords}
	action, information
\end{keywords}

\section{Introduction}
Observers (e.g.\ biological/silicon brains, sensory organs, measuring instruments) as part of their environment, do not have enough capacity (e.g.\ neurons, transistors) to store their observations.
To overcome this, part of their capacity could be used to learn and operate languages (natural, programming, computational, mathematical etc.).
Languages allow the observers to describe their observations, to make sense of the world and/or communicate.
However, this conciseness that languages provide comes with the cost of decreased observation accuracy.
For example, the brain internally reconstructs (describes) sensory observations with accuracy constrained by the physiology of each sensory organ (e.g.\ the human eye cannot distinguish objects less than a tenth of a millimeter).
Similarly, biological organisms may describe concepts or thoughts with varying levels of accuracy, constrained by the expressiveness of the language used within an environment.

Action is a functional that receives the history of a system as input and outputs a number.
In physics, the defining property of action is that it is minimized for the history that nature chose to follow (e.g.\ the specific path that an object followed when a force is applied to it).
The system for which we will define action is data (e.g.\ a set of numbers such as vectors) and the history is the description of their change; how they `traversed' from their initial to their final values.
Action has dimensions of energy \(\times\) time, however in our definition we use description length as a hardware-independent measure of time.
In other words, action is proportional to the quantity (energy) and duration (description length) of the change of the data. 
In the physical world, an object followed the one optimal path over the infinite possible paths, which can be found by minimizing the action (i.e.\ principle of least/stationary action~\cite{siburg2004principle}).
However data have multiple descriptions, which poses the following questions: 
\begin{itemize}
	\item Is there an optimal description for some data? If yes then, how can we approximate it?
	\item What does the value of the action mean?
\end{itemize}
We argue that action is the quantity of information of the data that is `perceived' by the description (viewed as an observer).

\section{Formal definition of Action}
Let \(x \in \mathbb{R}^n\) some data and \(d_{0 \rightarrow x} \in D\) a description of the traversal from \(0\) to \(x\), where \(D\) a description language.
We define action \(\mathcal{S}\) of \(x\) w.r.t.\ \(d_{0 \rightarrow x}\) as:
\[\mathcal{S} = \big\| x \big\| \big\| d_{0 \rightarrow x} \big\|\]
, where \(\big\| \cdot \big\|\) denotes the norm.
Calculating \(\big\| x \big\|\) is straightforward, however calculating \(\big\| d_{0 \rightarrow x} \big\|\) depends on \(d_{0 \rightarrow x}\), e.g.\ choosing a neural network as \(d_{0 \rightarrow x}\) we could have the following:
\[d_{0 \rightarrow x} = W + a + c\]
,where \(W\) is the number of weights, \(a\) the number of activations when fed with \(x\) and \(c\) a constant that is the description length of the neural network architecture. 
Let \(\hat{x}\) an approximation of \(x\), then \(\mathcal{S}\) becomes:
\[\mathcal{S} = \big\| \hat{x} \big\| \big\| d_{0 \rightarrow \hat{x}} \big\| + \big\| x - \hat{x} \big\| \big\| d_{\hat{x} \rightarrow x} \big\|\]
, where \(d_{0 \rightarrow \hat{x}}\) is the description for \(\hat{x}\), \(\big\| x - \hat{x} \big\|\) the energy of the loss and \(\big\| d_{\hat{x} \rightarrow x} \big\|\) the description length of the loss which could be considered as equal to the length of \(x\) in the worst case.
This formulation allows us to trade description length with energy and get closer to the minima of \(\mathcal{S}\).

\section{Discussion}
We can view \(\mathcal{S}\) as a deterministic metric of the informational content of \(x\) (observation) that is perceived by \(d_{0 \rightarrow \hat{x}}\) (observer); the lower the value of \(\mathcal{S}\), the less information (or less random) is perceived by \(d_{0 \rightarrow \hat{x}}\). 
According to previous literature, information is traditionally quantified as proportional to the degree of surprise that is contained in the data~\cite{shannon1948mathematical}.
However, this requires assuming the probability distribution of the data generator.
Kolmogorov-based information (complexity) of some data is defined as the minimum description length, but this is uncomputable~\cite{li2008introduction} and requires an exact description of the data.
\(\mathcal{S}\) can be seen as a generalization of Kolmogorov-based information; if we consider \(\hat{x} = x\), then the general problem of minimizing \(\mathcal{S}\) is equivalent with computing Kolmogorov complexity.
\(\mathcal{S}\) also differs from algorithmic information theory~\cite{chaitin1977algorithmic}, since it does not depend on the concept of probability. 
A question arises: `Is probabilistic information useful to us because of the limited brain capacity? '
Moreover, independent of the chosen description language, having a limited capacity only gets you so far to approximate an action minima of some data.
Thus there are data for which an action minima cannot be reached with a specific capacity; they will forever be out of reach of understanding and viewed as noise independent of the description we choose to describe them.
Maybe that is why we need to aggregate due to limited capacity.
Practically, \(\mathcal{S}\) minimization can be used for model selection in unsupervised problems and does not require an exact reconstruction like the Minimum Description Length (MDL)~\cite{rissanen1983universal}.
From the same point of view, \(\mathcal{S}\) could be used as a quantification of the bias-variance trade-off. 
\(\mathcal{S}\) minimization could be used to answer the question: `Why do we create and use languages? ': `Because languages reduce the \(\mathcal{S}\) of our observations'.
Lastly, \(\mathcal{S}\) is a lossy compression metric and could be viewed as an intelligence metric, which is in contrast with previous literature that assumes intelligence is lossless compression~\cite{hutter2012human}.

\section{Conclusion}
By defining \(\mathcal{S}\) for data and their descriptions we conclude that minimizing \(\mathcal{S}\) provides us with accurate and concise descriptions of data in the same way that action from physics shows as the path that nature chose to take.

\bibliography{ms.bib}

\end{document}
